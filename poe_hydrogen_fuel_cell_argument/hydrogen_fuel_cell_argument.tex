\documentclass[stu]{apa7}

\usepackage[american]{babel}

\usepackage{csquotes}
\usepackage[style=apa,sortcites=true,sorting=nyt,backend=biber]{biblatex}


\DeclareLanguageMapping{american}{american-apa}
\addbibresource{sources.bib}

\hypersetup{
    colorlinks=true,
    citecolor=black,
    urlcolor=blue,
}


\title{\textbf{The Argument For Hydrogen Fuel Cells}}
\author{Wyatt Geckle}
\affiliation{Project Lead the Way Engineering}
\course{Principles Of Engineering}
\professor{}  %confidential information. compiler will throw error, just press enter and ignore
\duedate{May 19, 2021}

\begin{document}

    \maketitle
    
    Our society will have a severe energy and environmental crisis unless we advance alternate
    sources of energy and energy storage into the public.  One such form of energy storage is
    the hydrogen fuel cell.  Hydrogen fuel cells are forms of energy storage where, according
    to Lucia, ``the chemical energy is directly converted into electrical energy''
    \parencite{lucia2014overview}.  Their main use in the current year is in electric vehicles,
    where they directly compete with lithium ion batteries.  However, although there is already
    a significant investment in other forms of energy storage, hydrogen fuel cells are worth
    investing time and resources into because of their significant efficiency advantages over
    batteries and decreased environmental impact compared to fossil fuels.
    
    
    Although there has already been substantial research regarding other forms of energy storage,
    the significant disadvantages of them have been well documented as a result.  Batteries are
    defined as devices containing cells or series of cells storing chemical energy that is
    converted to electric power.  The most popular reusable battery, the lithium ion, is limited
    by its physical structure, as the power it produces is dependent on the number of cells in the
    battery.  Therefore, as the energy capacity of a lithium ion cell increases, its mass
    proportionally increases.  Due to this, battery-powered vehicles experience weight compounding
    issues as their battery capacities increase.  According to Tsakiris, ``Each kilogram of battery
    weight to increase range requires extra structural weight, higher torque motor, heavier brakes,
    and in turn more batteries to carry the extra mass'' \parencite{tsakirisanalysis}.  Therefore,
    vehicles that use batteries have effeciency disadvantages compared to those that use
    alternative forms of energy storage.  On the other hand, forms of energy storage that involve
    non-renewable resources such as oil are more efficient, but they are limited and have
    severe environmental consequences.  Fossil fuels contribute to the largest amount of greenhouse
    gas emission in the United States, as cited by the United States Environmental Protection
    Agency.  Greenhouse gas emission is a leading cause in the recent global warming issue, and
    can prevent safe respiration in environments where emissions are excessive such as Bejing,
    China.  By using surplus amounts of these non-renewable energy sources, humanity is cutting
    down on their reserves for potential future energy crisis, and they're actively destroying
    the environment.  Even with the knowledge we have on these energy sources, we need an
    alternative that will both have efficiency and leave a smaller impact on the environment.
    
    
    Hydrogen fuel cells have significant efficiency advantages compared to lithium ion batteries
    in the automotive industry.  While batteries cause significant weight compounding issues with
    motor vehicles, hydrogen fuel cells avoid this issue.  The energy density of a hydrogen fuel
    cell is much smaller than a battery, and the proportional mass increase with respect to the
    energy capacity is much smaller.  Thomas states, ``the extra weight to increase the range of
    the fuel cell EV is negligible, while the battery EV weight escalates dramatically for ranges
    greater than 100 to 150 miles due to weight compounding.'' \parencite{thomas2009comparison}.
    According to Figure 4 in the article, the weight of a battery-powered vehicle increases
    exponentially with respect to the range, while the weight of a fuel cell-powered vehicle
    increases negligibly.  The exponential increase in the battery-powered vehicle's weight is
    due to the significant increase in battery mass, while the weight of the cell-powered vehicle
    increases negligibly because of the significantly decreased cell mass.  Therefore, the
    efficiency of hydrogen fuel cells is significantly greater than that of batteries, and
    as such they should be invested in as viable solutions to solve the issues batteries
    currently have.
    
    
    While fossil fuels have severe environmental implications with their carbon emissions, hydrogen
    fuel cells have a significantly reduced environmental footprint.  Fossil fuels are not only a
    limited resource, but using them up has severe environmental costs through carbon emissions.
    Despite this, humanity has been using an exponentially increasing amount of these energy
    resources as time as progressed.  According to Ritchie and Roser, ``Emissions have continued to
    grow rapidly; we now emit over 36 billion tonnes each year'' \parencite{ritchie2020emissions}.
    If we continue down this route, the effects of the increasing emissions will become
    irreversible, and it may create a potential energy crisis when the limited resources are
    depleted.  Hydrogen fuel cells can solve both of these issues, as they have a significantly
    decreased footprint compared to fossil fuels, and they're reusable. Fuel cell vehicles are able
    to travel similar ranges compared to gasoline vehicles while producing only about half the
    greenhouse gas emissions. Thomas claims that ``the hydrogen FCV would immediately cut GHG
    emissions by more than 50\% compared to regular cars'' \parencite{thomas2009comparison}.
    Especially since fossil fuel emissions rise exponentially with each passing year, the ability
    to drop emissions by half would significantly improve environmental prospects.  As a result,
    hydrogen fuel cells should be considered as an alternative fuel source that will
    substantially decrease humanity's environmental impact while maintaining the efficiency
    that is expected from a fuel source. 
    
    
    While it is true that hydrogen fuel cells don't have universal applications like batteries do,
    as they will likely never become relevant in portable devices, their specific advantages
    are highly useful in the industries they benefit.  Therefore, fuel cells are worth the time
    and research investment to increase energy reusability, decrease environmental impact, and
    maintain high efficiency in vehicles.
    
    
    \printbibliography

\end{document}
